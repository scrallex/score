% Reliability‑Gated Recurrence Detection in a Structural Feature Manifold
%
% This document is a preliminary manuscript describing the methodology,
% experimental design and results associated with the reliability‑gated
% recurrence detector described in the accompanying software.  It
% synthesises findings from a series of backtests across multiple
% foreign‑exchange instruments.  The intention of this draft is to
% provide a clear narrative for future peer review and to serve as the
% basis for a more polished submission.

\documentclass[11pt]{article}
\usepackage[margin=1in]{geometry}
\usepackage{amsmath,amssymb}
\usepackage{graphicx}
\usepackage{hyperref}
\hypersetup{colorlinks=true,linkcolor=blue,citecolor=blue,urlcolor=blue}

\title{Reliability‑Gated Recurrence Detection in a Structural Feature Manifold}
\author{Your Name}
\date{\today}

\begin{document}

\maketitle

\begin{abstract}
This paper proposes and evaluates a detection framework for
identifying temporary regimes in financial time series.  We embed
OHLCV (open, high, low, close, volume) candle sequences in a
five–dimensional structural feature manifold derived from the
quantised functional hierarchy (QFH) and structural trading manifold
(STM).  Each bar yields a feature vector $(c,s,H,\rho,\lambda)$
representing coherence, stability, entropy, rupture density and
hazard; these variables are bounded in $[0,1]$ and capture the
dynamical complexity of the series.  A signature bucket is defined by
\emph{discretising} $(c,s,H)$ to two decimal places; the detector admits a
window when its signature has been observed at least three times in
the recent history and its hazard $\lambda$ falls below a preset
threshold.  We refer to this event as a ``reliable recurrence''
although it is distinct from an acoustic echo.  We test this
framework on eight FX and precious metal pairs, demonstrating
statistically significant positive expectancy under momentum
configurations and uniformly negative results under mean–reversion,
thus validating the hypothesis that alpha correlates with temporary
regimes defined in feature space.
\end{abstract}

\section{Introduction}
Financial markets exhibit transient phases of predictability and
structural repetition.  Capturing these regimes requires methods
beyond linear indicators or classic technical analysis.  We propose a
``reliability‑gated recurrence'' detector operating in a structural
feature manifold.  Rather than forecasting prices, we perform
hypothesis testing in feature space: does the current window belong
to a previously observed stable state and is it sufficiently
``reliable''?  When both conditions hold, we enter a trade; if the
pattern is novel or unstable, we stand aside.  This paper formalises
the detector and presents backtesting results across multiple
currency pairs.

\section{Method}

\subsection{Structural feature manifold and signal generation}
We derive our features from the STM/QFH pipeline described in
\cite{reference_stm}.  Each one–minute candle is processed by the
native QFH kernel to produce five bounded metrics: coherence $c$,
stability $s$, entropy $H$, rupture density $\rho$ and hazard
$\lambda$.  Coherence quantifies the alignment of price vectors; stability
measures irreversibility; entropy captures state diversity; rupture
density counts transitions; and hazard represents the propensity for
structural collapse.  We define a signature
$\sigma=(\operatorname{round}(c,\epsilon),\operatorname{round}(s,\epsilon),\operatorname{round}(H,\epsilon))$
with precision $\epsilon=0.01$.  A repetition statistic $R_t$ counts
the number of times $\sigma$ has appeared in the past $W$ minutes.

\subsection{Reliability gate and trading rule}
At time $t$ the detector applies a simple rule: if
$R_t\ge R_{\min}$ (we use $R_{\min}=3$ throughout) and
$\lambda_t \le \lambda_{\max}$, an admission occurs.  In that case
we open a trade in the direction of the most recent momentum
(``momentum'' regime) or against it (``mean–reversion'' regime).
Entries are filtered by session masks corresponding to regional
market hours and by instrument–specific thresholds on $c$, $s$ and
$H$ to remove noisy signatures.  Exits are governed by either a
multiple of the average true range (ATR) or a fixed basis point
(BPS) target, with horizons of 40–60 minutes.  Position size is
constant so results translate directly to expected return per trade.

\section{Experiment design}

\subsection{Data and sessions}
We sourced 45‐day and 90‐day windows of one–minute OHLCV data for
EUR/USD, GBP/USD, USD/JPY, USD/CHF, AUD/USD, NZD/USD, USD/CAD and
XAU/USD via the OANDA API.  Per–instrument sessions reflect
primary trading hours: London for EUR/USD and USD/CHF; extended
London/New York for GBP/USD and USD/CAD; Tokyo for USD/JPY; Sydney
for AUD/NZD; and a 10:00–20:00~Z window overlapping COMEX for
XAU/USD.  We enforced minimum coherence, stability and maximum
entropy thresholds per instrument to exclude highly unstable bars.
These settings were tuned in preliminary ``scout'' runs until each
leg produced at least 25 trades with positive Sharpe.  The YAML
configuration for the full sweep is publicly available in our
repository\cite{reference_config45} and shows the data paths,
session windows and threshold parameters per instrument\cite{reference_config90}.

\subsection{Parameter grid}
We tested two directions (momentum/mean–reversion), two values of
$R_{\min}$ (2 and 3), three hazard caps (0.25, 0.35, 0.45) and
three exit horizons (20, 40, 60 minutes), yielding 36 parameter
combinations.  During initial scouting we restricted the grid to
$(R_{\min},\lambda_{\max},h)\in\{(3,0.25,40),(3,0.35,40),(3,0.35,60)\}$
with bootstrap iterations reduced to 100 to speed up tuning.  For
the full sweep we used bootstrap resampling with 200 iterations for
reporting and 750 iterations for final confirmation.  This
approximation produces tight confidence intervals on performance
statistics without prohibitive runtimes.

\subsection{Evaluation metrics}
We compute expected return per trade (basis points), win rate,
payoff ratio (average win size divided by average loss size), Sharpe
ratio, Calmar ratio (return divided by maximum drawdown), profit
factor and mean time in trade.  A baseline hold‐until‐horizon
strategy serves as a control to compute alpha.  Statistical
significance is assessed by bootstrap: we resample the trade P\&L
series 200–750 times; $p$–values below 0.05 indicate that the
observed mean is unlikely under the null.  Hazard‐admission
calibration curves and lead‐time histograms provide diagnostic
views of the detector’s behaviour.

\section{Results}

\subsection{Anchor sweep (45 days)}
Our 45‐day sweep across eight instruments (36 parameter
combinations) revealed a consistent profitable regime.  Momentum
trading with $R_{\min}=3$, exit horizon of 40 minutes and
$\lambda_{\max}\in\{0.25,0.35,0.45\}$ dominated the results.
A representative configuration ($\lambda_{\max}=0.35$) yielded 259
trades across the portfolio with an average return of 3.38 bps per
trade, Sharpe ratio 2.50, Calmar ratio 4.23, profit factor 1.57 and
a bootstrap $p$–value of 0.004\cite{reference_anchor}.  Mean–reversion
configurations with the same thresholds were uniformly negative
across all legs, serving as a control and reinforcing that the alpha
is derived from momentum alignment rather than noise\cite{reference_anchor}.

Breaking down the anchor sweep by instrument, NZD/USD was the star
performer (+9.38 bps, Sharpe 2.25), followed by GBP/USD (+4.32 bps,
Sharpe 1.44), USD/CHF (+3.22 bps, Sharpe 1.00), USD/CAD (+2.69 bps,
Sharpe 1.41), EUR/USD (+1.31 bps, Sharpe 0.56) and AUD/USD (+1.32 bps,
Sharpe 0.60).  USD/JPY returned −0.54 bps (Sharpe −0.25) and XAU/USD
was flat.  Hazard–admission curves were monotone and lead‐time
histograms clustered between 25–45 bars, validating the 40‐minute
exit horizon\cite{reference_anchor}.

\subsection{Robustness sweep (90 days)}
Extending the window to 90 days and adjusting session windows for
USD/JPY and XAU/USD produced 424 trades with 0.87 bps expected
return, Sharpe 1.10 and profit factor 1.16.  EUR/USD and
AUD/USD flattened slightly but remained near zero after tightening
coherence and stability floors.  The largest improvements were
observed in USD/JPY (−0.54 bps to +2.25 bps) and XAU/USD (+0.13 bps
to +7.9 bps).  The hazard threshold remained non–sensitive; all
winning configurations used $R_{\min}=3$ and horizons of 40–60 minutes.

\section{Discussion}
Several conclusions emerge from our experiments:
\begin{itemize}
  \item \textbf{Reliability gating works.}  Requiring at least three
    recurrences and imposing a hazard cap yields significant positive
    expectancy across multiple instruments.  Without the recurrence
    requirement or by flipping direction to mean–reversion, alpha
    disappears\cite{reference_discussion}.
  \item \textbf{Hazard insensitivity.}  The detection rule is robust
    across $\lambda_{\max}$ values between 0.25 and 0.45; this
    stability suggests that the gating mechanism is not overtuned and
    may generalise to unseen data.
  \item \textbf{Session and threshold sensitivity.}  Instruments
    respond differently to session windows and noise filters.
    USD/JPY and XAU/USD only become profitable when their sessions are
    contracted to Tokyo and COMEX and their coherence and stability
    thresholds raised; EUR/USD and AUD/USD require tuning on the
    longer sample\cite{reference_discussion}.
  \item \textbf{Control validation.}  Mean–reversion regimes provide
    a strong negative control; reversing the polarity under
    identical hazard and repetition thresholds results in negative
    expectancy on every instrument.
  \item \textbf{Limitations and next steps.}  Forty–five days
    remains a modest sample; we intend to extend to six months or
    beyond to test stability.  Adaptive hazard caps that vary with
    macro volatility and multi‐frequency manifolds are promising
    avenues.  Real‐time integration with a portfolio risk engine and
    Valkey persistence will enable continuous monitoring\cite{reference_discussion}.
\end{itemize}

\section{Conclusion}
We present a reliability–gated recurrence detector for identifying
temporary trading regimes.  By embedding candle sequences into a
structural feature manifold and enforcing both repetition and hazard
constraints, the detector isolates market states with statistically
positive expectancy.  Backtests across eight FX pairs and gold over
45 and 90 days confirm the efficacy of the method, with
multi–instrument momentum configurations producing Sharpe ratios
above 2.  Mean–reversion variants fail, underscoring the
importance of alignment with momentum in this feature space.  Future
work will explore longer horizons, adaptive thresholds and live
deployment.

\section*{Acknowledgements}
We thank the engineering team for providing access to the STM/QFH
codebase and the data ingestion pipeline.  All experiments were run
using the backtester and reporting suite.  The YAML configurations
for reproducing this study are included in the repository.

\begin{thebibliography}{9}
\bibitem{reference_stm} Documentation and methods for the structural
  trading manifold (STM) and quantised functional hierarchy (QFH) are
  described in the project documentation at
  \url{https://github.com/SepDynamics/spt/tree/main/docs}.

\bibitem{reference_config45} YAML configuration for the 45‐day
  multi‐instrument sweep detailing data sources, sessions and
  thresholds.  Source: internal repository.

\bibitem{reference_config90} YAML configuration for the 90‐day
  multi‐instrument sweep with adjusted sessions and thresholds.
  Source: internal repository.

\bibitem{reference_anchor} Section~3.1 of the draft paper reports the
  anchor sweep results: 259 trades, +3.38 bps, Sharpe 2.50 and
  evidence that mean–reversion is negative across legs.

\bibitem{reference_discussion} Section~4 of the draft paper summarises the
  discussion points: reliability gating, hazard insensitivity,
  session sensitivity, control validation, and next steps.
\end{thebibliography}

\end{document}